% LaTeX Table Template for Hippopotamus Optimization Research Results
% This template is used by LatexTableGenerator.java to create publication-ready tables

% ============================================================================
% Basic Table Template
% ============================================================================
\begin{table}[!htbp]
\centering
\caption{${caption}}
\label{tab:${label}}
\resizebox{\textwidth}{!}{%
\begin{tabular}{${columns}}
\toprule
${header}
\midrule
${content}
\bottomrule
\end{tabular}%
}
${footnotes}
\end{table}

% ============================================================================
% Algorithm Comparison Table Template
% ============================================================================
\begin{table}[!htbp]
\centering
\caption{Performance comparison of VM placement algorithms across different metrics}
\label{tab:algorithm_comparison}
\resizebox{\textwidth}{!}{%
\begin{tabular}{lcccccc}
\toprule
\multirow{2}{*}{\textbf{Algorithm}} & \multicolumn{3}{c}{\textbf{Resource Utilization (\%)}} & \multirow{2}{*}{\textbf{Power (kWh)}} & \multirow{2}{*}{\textbf{SLA Violations}} & \multirow{2}{*}{\textbf{Execution Time (s)}} \\
\cmidrule(lr){2-4}
 & \textbf{CPU} & \textbf{Memory} & \textbf{Bandwidth} & & & \\
\midrule
${algorithm_rows}
\bottomrule
\end{tabular}%
}
\begin{tablenotes}
\small
\item Values represent mean ± standard deviation over ${num_runs} independent runs
\item Best values are highlighted in \textbf{bold}
\item ${significance_note}
\end{tablenotes}
\end{table}

% ============================================================================
% Statistical Test Results Table Template
% ============================================================================
\begin{table}[!htbp]
\centering
\caption{Statistical significance test results (p-values) for pairwise algorithm comparisons}
\label{tab:statistical_tests}
\begin{tabular}{lccccccc}
\toprule
\textbf{Metric} & \textbf{HO vs FF} & \textbf{HO vs BF} & \textbf{HO vs GA} & \textbf{HO vs PSO} & \textbf{HO vs ACO} & \textbf{HO vs Random} & \textbf{Test Used} \\
\midrule
${statistical_rows}
\bottomrule
\end{tabular}
\begin{tablenotes}
\small
\item Significance level: $\alpha = ${alpha_level}$
\item p-values < 0.05 are marked with *, p < 0.01 with **, p < 0.001 with ***
\item HO: Hippopotamus Optimization, FF: First Fit, BF: Best Fit, GA: Genetic Algorithm
\item PSO: Particle Swarm Optimization, ACO: Ant Colony Optimization
\end{tablenotes}
\end{table}

% ============================================================================
% Scalability Analysis Table Template
% ============================================================================
\begin{table}[!htbp]
\centering
\caption{Scalability analysis: Algorithm performance with increasing problem size}
\label{tab:scalability_analysis}
\resizebox{\textwidth}{!}{%
\begin{tabular}{lrrrrrr}
\toprule
\multirow{2}{*}{\textbf{Algorithm}} & \multicolumn{6}{c}{\textbf{Number of VMs / Execution Time (seconds)}} \\
\cmidrule(lr){2-7}
 & \textbf{100} & \textbf{500} & \textbf{1000} & \textbf{2000} & \textbf{5000} & \textbf{Time Complexity} \\
\midrule
${scalability_rows}
\bottomrule
\end{tabular}%
}
\end{table}

% ============================================================================
% Parameter Sensitivity Table Template
% ============================================================================
\begin{table}[!htbp]
\centering
\caption{Parameter sensitivity analysis for Hippopotamus Optimization algorithm}
\label{tab:parameter_sensitivity}
\begin{tabular}{lccccc}
\toprule
\textbf{Parameter} & \textbf{Range Tested} & \textbf{Optimal Value} & \textbf{Impact on Performance} & \textbf{Sensitivity Index} & \textbf{p-value} \\
\midrule
${parameter_rows}
\bottomrule
\end{tabular}
\begin{tablenotes}
\small
\item Impact scale: Low (< 5\%), Medium (5-15\%), High (> 15\%)
\item Sensitivity index calculated using Sobol's method
\end{tablenotes}
\end{table}

% ============================================================================
% Dataset Comparison Table Template
% ============================================================================
\begin{table}[!htbp]
\centering
\caption{Algorithm performance across different real-world datasets}
\label{tab:dataset_comparison}
\resizebox{\textwidth}{!}{%
\begin{tabular}{llccccr}
\toprule
\textbf{Dataset} & \textbf{Algorithm} & \textbf{Resource Util. (\%)} & \textbf{Power (kWh)} & \textbf{SLA Viol. (\%)} & \textbf{Migrations} & \textbf{Total Cost (\$)} \\
\midrule
${dataset_rows}
\bottomrule
\end{tabular}%
}
\begin{tablenotes}
\small
\item Google: Google cluster traces, Azure: Azure VM traces, Synthetic: Generated workloads
\item All values averaged over entire trace duration
\end{tablenotes}
\end{table}

% ============================================================================
% Convergence Analysis Table Template
% ============================================================================
\begin{table}[!htbp]
\centering
\caption{Convergence characteristics of optimization algorithms}
\label{tab:convergence_analysis}
\begin{tabular}{lccccc}
\toprule
\textbf{Algorithm} & \textbf{Iterations to 90\%} & \textbf{Iterations to 95\%} & \textbf{Final Fitness} & \textbf{Std. Dev.} & \textbf{Success Rate (\%)} \\
\midrule
${convergence_rows}
\bottomrule
\end{tabular}
\begin{tablenotes}
\small
\item Convergence measured as percentage of final fitness value achieved
\item Success rate: percentage of runs achieving target fitness threshold
\end{tablenotes}
\end{table}

% ============================================================================
% Multi-objective Results Table Template
% ============================================================================
\begin{table}[!htbp]
\centering
\caption{Multi-objective optimization results: Pareto front solutions}
\label{tab:multiobjective_results}
\begin{tabular}{lcccccc}
\toprule
\textbf{Solution} & \textbf{Resource Util.} & \textbf{Power} & \textbf{SLA Viol.} & \textbf{Load Balance} & \textbf{Migrations} & \textbf{Hypervolume} \\
\midrule
${pareto_rows}
\bottomrule
\end{tabular}
\begin{tablenotes}
\small
\item All objectives normalized to [0, 1] range
\item Hypervolume indicator calculated with reference point (1.1, 1.1, 1.1, 1.1, 1.1)
\end{tablenotes}
\end{table}

% ============================================================================
% Summary Statistics Table Template
% ============================================================================
\begin{table}[!htbp]
\centering
\caption{Summary statistics for ${metric_name} across all experiments}
\label{tab:summary_stats_${metric_label}}
\begin{tabular}{lrrrrrrr}
\toprule
\textbf{Algorithm} & \textbf{Mean} & \textbf{Median} & \textbf{Std. Dev.} & \textbf{Min} & \textbf{Max} & \textbf{CI (95\%)} & \textbf{CV (\%)} \\
\midrule
${summary_rows}
\bottomrule
\end{tabular}
\begin{tablenotes}
\small
\item CI: Confidence Interval, CV: Coefficient of Variation
\item Based on ${num_experiments} experiments with ${num_replications} replications each
\end{tablenotes}
\end{table}

% ============================================================================
% Resource Utilization Breakdown Table Template
% ============================================================================
\begin{table}[!htbp]
\centering
\caption{Detailed resource utilization breakdown by algorithm}
\label{tab:resource_breakdown}
\resizebox{\textwidth}{!}{%
\begin{tabular}{lcccccccc}
\toprule
\multirow{2}{*}{\textbf{Algorithm}} & \multicolumn{4}{c}{\textbf{Average Utilization (\%)}} & \multicolumn{4}{c}{\textbf{Peak Utilization (\%)}} \\
\cmidrule(lr){2-5} \cmidrule(lr){6-9}
 & \textbf{CPU} & \textbf{Memory} & \textbf{Storage} & \textbf{Network} & \textbf{CPU} & \textbf{Memory} & \textbf{Storage} & \textbf{Network} \\
\midrule
${utilization_rows}
\bottomrule
\end{tabular}%
}
\end{table}

% ============================================================================
% Time Complexity Analysis Table Template
% ============================================================================
\begin{table}[!htbp]
\centering
\caption{Time and space complexity analysis of VM placement algorithms}
\label{tab:complexity_analysis}
\begin{tabular}{lccc}
\toprule
\textbf{Algorithm} & \textbf{Time Complexity} & \textbf{Space Complexity} & \textbf{Empirical Growth Rate} \\
\midrule
Hippopotamus Optimization & $O(n \cdot p \cdot i)$ & $O(n \cdot p)$ & ${ho_growth} \\
First Fit & $O(n \cdot m)$ & $O(1)$ & ${ff_growth} \\
Best Fit & $O(n \cdot m)$ & $O(1)$ & ${bf_growth} \\
Genetic Algorithm & $O(n \cdot p \cdot g)$ & $O(n \cdot p)$ & ${ga_growth} \\
Particle Swarm & $O(n \cdot s \cdot i)$ & $O(n \cdot s)$ & ${pso_growth} \\
Ant Colony & $O(n^2 \cdot a \cdot i)$ & $O(n^2)$ & ${aco_growth} \\
\bottomrule
\end{tabular}
\begin{tablenotes}
\small
\item $n$: number of VMs, $m$: number of hosts, $p$: population size
\item $i$: iterations, $g$: generations, $s$: swarm size, $a$: number of ants
\item Empirical growth rate determined by regression analysis
\end{tablenotes}
\end{table}

% ============================================================================
% Custom Table Builder Macros
% ============================================================================

% Macro for creating simple comparison tables
\newcommand{\simplecomparisontable}[4]{%
\begin{table}[!htbp]
\centering
\caption{#1}
\label{tab:#2}
\begin{tabular}{#3}
\toprule
#4
\bottomrule
\end{tabular}
\end{table}
}

% Macro for highlighted best values
\newcommand{\best}[1]{\textbf{#1}}

% Macro for statistical significance markers
\newcommand{\sig}[1]{#1$^{*}$}
\newcommand{\sigsig}[1]{#1$^{**}$}
\newcommand{\sigsigsig}[1]{#1$^{***}$}

% Macro for confidence intervals
\newcommand{\ci}[2]{#1 $\pm$ #2}

% ============================================================================
% Notes on Template Usage
% ============================================================================
% This template file is used by LatexTableGenerator.java
% Variables in ${} format are replaced with actual values
% Tables are designed to be publication-ready with proper formatting
% All tables follow standard academic formatting conventions
% Tables can be customized by modifying the template or using custom builders